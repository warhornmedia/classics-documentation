% Options for packages loaded elsewhere
\PassOptionsToPackage{unicode}{hyperref}
\PassOptionsToPackage{hyphens}{url}
%
\documentclass[
]{book}
\usepackage{lmodern}
\usepackage{amssymb,amsmath}
\usepackage{ifxetex,ifluatex}
\ifnum 0\ifxetex 1\fi\ifluatex 1\fi=0 % if pdftex
  \usepackage[T1]{fontenc}
  \usepackage[utf8]{inputenc}
  \usepackage{textcomp} % provide euro and other symbols
\else % if luatex or xetex
  \usepackage{unicode-math}
  \defaultfontfeatures{Scale=MatchLowercase}
  \defaultfontfeatures[\rmfamily]{Ligatures=TeX,Scale=1}
\fi
% Use upquote if available, for straight quotes in verbatim environments
\IfFileExists{upquote.sty}{\usepackage{upquote}}{}
\IfFileExists{microtype.sty}{% use microtype if available
  \usepackage[]{microtype}
  \UseMicrotypeSet[protrusion]{basicmath} % disable protrusion for tt fonts
}{}
\makeatletter
\@ifundefined{KOMAClassName}{% if non-KOMA class
  \IfFileExists{parskip.sty}{%
    \usepackage{parskip}
  }{% else
    \setlength{\parindent}{0pt}
    \setlength{\parskip}{6pt plus 2pt minus 1pt}}
}{% if KOMA class
  \KOMAoptions{parskip=half}}
\makeatother
\usepackage{xcolor}
\IfFileExists{xurl.sty}{\usepackage{xurl}}{} % add URL line breaks if available
\IfFileExists{bookmark.sty}{\usepackage{bookmark}}{\usepackage{hyperref}}
\hypersetup{
  pdftitle={Warhorn Media Style Guide},
  hidelinks,
  pdfcreator={LaTeX via pandoc}}
\urlstyle{same} % disable monospaced font for URLs
\usepackage{color}
\usepackage{fancyvrb}
\newcommand{\VerbBar}{|}
\newcommand{\VERB}{\Verb[commandchars=\\\{\}]}
\DefineVerbatimEnvironment{Highlighting}{Verbatim}{commandchars=\\\{\}}
% Add ',fontsize=\small' for more characters per line
\usepackage{framed}
\definecolor{shadecolor}{RGB}{248,248,248}
\newenvironment{Shaded}{\begin{snugshade}}{\end{snugshade}}
\newcommand{\AlertTok}[1]{\textcolor[rgb]{0.94,0.16,0.16}{#1}}
\newcommand{\AnnotationTok}[1]{\textcolor[rgb]{0.56,0.35,0.01}{\textbf{\textit{#1}}}}
\newcommand{\AttributeTok}[1]{\textcolor[rgb]{0.77,0.63,0.00}{#1}}
\newcommand{\BaseNTok}[1]{\textcolor[rgb]{0.00,0.00,0.81}{#1}}
\newcommand{\BuiltInTok}[1]{#1}
\newcommand{\CharTok}[1]{\textcolor[rgb]{0.31,0.60,0.02}{#1}}
\newcommand{\CommentTok}[1]{\textcolor[rgb]{0.56,0.35,0.01}{\textit{#1}}}
\newcommand{\CommentVarTok}[1]{\textcolor[rgb]{0.56,0.35,0.01}{\textbf{\textit{#1}}}}
\newcommand{\ConstantTok}[1]{\textcolor[rgb]{0.00,0.00,0.00}{#1}}
\newcommand{\ControlFlowTok}[1]{\textcolor[rgb]{0.13,0.29,0.53}{\textbf{#1}}}
\newcommand{\DataTypeTok}[1]{\textcolor[rgb]{0.13,0.29,0.53}{#1}}
\newcommand{\DecValTok}[1]{\textcolor[rgb]{0.00,0.00,0.81}{#1}}
\newcommand{\DocumentationTok}[1]{\textcolor[rgb]{0.56,0.35,0.01}{\textbf{\textit{#1}}}}
\newcommand{\ErrorTok}[1]{\textcolor[rgb]{0.64,0.00,0.00}{\textbf{#1}}}
\newcommand{\ExtensionTok}[1]{#1}
\newcommand{\FloatTok}[1]{\textcolor[rgb]{0.00,0.00,0.81}{#1}}
\newcommand{\FunctionTok}[1]{\textcolor[rgb]{0.00,0.00,0.00}{#1}}
\newcommand{\ImportTok}[1]{#1}
\newcommand{\InformationTok}[1]{\textcolor[rgb]{0.56,0.35,0.01}{\textbf{\textit{#1}}}}
\newcommand{\KeywordTok}[1]{\textcolor[rgb]{0.13,0.29,0.53}{\textbf{#1}}}
\newcommand{\NormalTok}[1]{#1}
\newcommand{\OperatorTok}[1]{\textcolor[rgb]{0.81,0.36,0.00}{\textbf{#1}}}
\newcommand{\OtherTok}[1]{\textcolor[rgb]{0.56,0.35,0.01}{#1}}
\newcommand{\PreprocessorTok}[1]{\textcolor[rgb]{0.56,0.35,0.01}{\textit{#1}}}
\newcommand{\RegionMarkerTok}[1]{#1}
\newcommand{\SpecialCharTok}[1]{\textcolor[rgb]{0.00,0.00,0.00}{#1}}
\newcommand{\SpecialStringTok}[1]{\textcolor[rgb]{0.31,0.60,0.02}{#1}}
\newcommand{\StringTok}[1]{\textcolor[rgb]{0.31,0.60,0.02}{#1}}
\newcommand{\VariableTok}[1]{\textcolor[rgb]{0.00,0.00,0.00}{#1}}
\newcommand{\VerbatimStringTok}[1]{\textcolor[rgb]{0.31,0.60,0.02}{#1}}
\newcommand{\WarningTok}[1]{\textcolor[rgb]{0.56,0.35,0.01}{\textbf{\textit{#1}}}}
\usepackage{longtable,booktabs}
% Correct order of tables after \paragraph or \subparagraph
\usepackage{etoolbox}
\makeatletter
\patchcmd\longtable{\par}{\if@noskipsec\mbox{}\fi\par}{}{}
\makeatother
% Allow footnotes in longtable head/foot
\IfFileExists{footnotehyper.sty}{\usepackage{footnotehyper}}{\usepackage{footnote}}
\makesavenoteenv{longtable}
\usepackage{graphicx}
\makeatletter
\def\maxwidth{\ifdim\Gin@nat@width>\linewidth\linewidth\else\Gin@nat@width\fi}
\def\maxheight{\ifdim\Gin@nat@height>\textheight\textheight\else\Gin@nat@height\fi}
\makeatother
% Scale images if necessary, so that they will not overflow the page
% margins by default, and it is still possible to overwrite the defaults
% using explicit options in \includegraphics[width, height, ...]{}
\setkeys{Gin}{width=\maxwidth,height=\maxheight,keepaspectratio}
% Set default figure placement to htbp
\makeatletter
\def\fps@figure{htbp}
\makeatother
\setlength{\emergencystretch}{3em} % prevent overfull lines
\providecommand{\tightlist}{%
  \setlength{\itemsep}{0pt}\setlength{\parskip}{0pt}}
\setcounter{secnumdepth}{5}
% DEFINE PHYSICAL DOCUMENT SETTINGS HD
% media settings
\usepackage[paperwidth=5.5in, paperheight=8.5in]{geometry}

\usepackage{booktabs}
\usepackage{amsthm}
\makeatletter
\def\thm@space@setup{%
  \thm@preskip=8pt plus 2pt minus 4pt
  \thm@postskip=\thm@preskip
}

\newenvironment{poetry}[0]{\par\leftskip=2em\rightskip=2em}{\par\medskip}

\newfontfamily\greekfont[Script=Greek]{LiberationSerif}

\makeatother

\frontmatter
\ifluatex
  \usepackage{selnolig}  % disable illegal ligatures
\fi
\usepackage[]{natbib}
\bibliographystyle{plainnat}

\title{Warhorn Media Style Guide}
\author{}
\date{\vspace{-2.5em}}

\begin{document}
\maketitle

\mainmatter

{
\setcounter{tocdepth}{1}
\tableofcontents
}
\hypertarget{preface}{%
\chapter*{Preface}\label{preface}}
\addcontentsline{toc}{chapter}{Preface}

This guide is meant to be a repository of information on both \emph{what} we have decided in terms of various style decisions, as well as \emph{how} to accomplish those decisions. Currently the documentation is limited to books, and in particular, those in the Warhorn Classics collection. However, it is possible that additional information will be added both for other kinds of books and for the WarhornMedia.com website.

Speaking of WarhornMedia.com\ldots{} post titles are to be in sentence caps. Consider it documented. :)

\hypertarget{warhorn-classics-books}{%
\chapter{Warhorn Classics Books}\label{warhorn-classics-books}}

Warhorn Classics uses Bookdown to create all its online versions of books. For technical questions about how to accomplish something that is not covered in this guide, \href{https://bookdown.org/yihui/bookdown/}{this book} will likely answer the question or put you on the path to finding out.

\hypertarget{book-structure}{%
\section{Book structure}\label{book-structure}}

\hypertarget{chapters-sections-and-more}{%
\subsection{Chapters, sections, and more}\label{chapters-sections-and-more}}

The primary structure in Warhorn Classics books is determined by various levels of headers, and it can easily be seen in the automatically generated table of contents.

By default, the top level of structure is called a ``chapter.'' However, this word can be changed in \_bookdown.yml if necessary for sermons or other types of works where ``chapter'' is not appropriate.\footnote{It is \href{https://bookdown.org/yihui/bookdown/markdown-extensions-by-bookdown.html\#special-headers}{also possible} to split a book into ``parts'' made up of multiple chapters, as well as add a special ``part'' called ``appendix.'' However, ebooks and Word documents will not include ``part'' information, so these should be avoided, if possible, or tested thoroughly, so you know what each file type does.}

The start of a chapter is specified in the text with a \# followed by the title of the chapter:

\begin{Shaded}
\begin{Highlighting}[]
\FunctionTok{\# A Long{-}expected chapter title}
\end{Highlighting}
\end{Shaded}

Chapters can be broken down further into sections and sub-sections, etc. using additional levels of headers and titles:

\begin{Shaded}
\begin{Highlighting}[]
\FunctionTok{\#\# This is a section.}

\FunctionTok{\#\#\# Here is a sub{-}section.}

\FunctionTok{\#\#\#\# And now a sub{-}sub{-}section}
\end{Highlighting}
\end{Shaded}

\hypertarget{numbering}{%
\subsubsection{Numbering}\label{numbering}}

Chapters, sections and subsections will all be automatically numbered, unless you exclude them from numbering by adding ``\{-\}'' to the end of the line. For example, generally the introduction is not numbered.

\begin{Shaded}
\begin{Highlighting}[]
\FunctionTok{\# Introduction \{{-}\}}

\NormalTok{Text of the introduction goes here...}
\end{Highlighting}
\end{Shaded}

Or perhaps the subsections in the book are not numbered:

\begin{Shaded}
\begin{Highlighting}[]
\FunctionTok{\# How to write a book}

\FunctionTok{\#\# Getting started}

\FunctionTok{\#\#\# Arranging your pencils \{{-}\}}
\end{Highlighting}
\end{Shaded}

\hypertarget{file-structure}{%
\subsection{File structure}\label{file-structure}}

\textbf{Note:} Supposedly the first file must begin with either a chapter (``\# Chapter name'') or a section (``\#\# Section title''). However, it appears that it must be a chapter, as starting with a section causes an error.

The online version of the book will be split up into separate files, not just one long web page. The split will happen at each new .rmd file.\footnote{This \href{https://bookdown.org/yihui/bookdown/html.html\#gitbook-style}{can be changed}, if necessary, to split by chapter or section, or even turned off completely.} Thus, in most cases a new .rmd file should be created for each chapter.

Each file must begin with the title of the chapter as outlined above. The various files will all be combined into a book, ordered by filename, so use numbers at the beginning of the files to indicate the order they should go in. Note that index.rmd will always come first, though, and will automatically include the Warhorn Classics cover page. Here is an example list of files in the order they will end up in the book.

\begin{verbatim}
index.Rmd
00-preface.Rmd
01-The-Sacraments-In-General.Rmd
02-The-Sacrament-of-Baptism.Rmd
03-The-Sacrament-of-the-Lords-Supper.Rmd
\end{verbatim}

\hypertarget{formatting}{%
\section{Formatting}\label{formatting}}

\hypertarget{overview}{%
\subsection{Overview}\label{overview}}

Formatting should not be used for structural elements such as headers or captions. Our templates will style those elements according to a standard design. Formatting should only be done where the formatting is essential to the text. For example, bold or italics that the author is using for emphasis should be included, whereas if a (sub)section heading is italicized in the source book, that is a question of design.

Formatting is accomplished by using R-markdown. Basically, you can use anything that \href{https://pandoc.org/MANUAL.html\#pandocs-markdown}{Pandoc supports}. Where Pandoc supports multiple options, we have generally chosen a specific method. If you want to
\#\#\# Italics

\emph{Italicized text} is indicated by surrounding it with single asterisks.

\begin{Shaded}
\begin{Highlighting}[]
\NormalTok{Here are *a couple* in italics.}
\end{Highlighting}
\end{Shaded}

\hypertarget{bold}{%
\subsection{Bold}\label{bold}}

\textbf{Bold text} is indicated by surrounding it with double asterisks.

\begin{Shaded}
\begin{Highlighting}[]
\NormalTok{Here are **a few words** in bold.}
\end{Highlighting}
\end{Shaded}

\hypertarget{small-caps}{%
\subsection{Small caps}\label{small-caps}}

I cannot think of any other circumstance where smallcaps should be used except the word \textsc{Lord}.

\begin{Shaded}
\begin{Highlighting}[]
\CommentTok{[}\OtherTok{Lord}\CommentTok{]}\NormalTok{\{.smallcaps\}}
\end{Highlighting}
\end{Shaded}

\hypertarget{centering-text}{%
\subsection{Centering text}\label{centering-text}}

I cannot think of a place where text would need to be centered in the text of a book.

\begin{Shaded}
\begin{Highlighting}[]
\NormalTok{::: \{.center\}}
\NormalTok{Republished by Warhorn Classics}
\NormalTok{:::}
\end{Highlighting}
\end{Shaded}

\hypertarget{special-characters}{%
\section{Special characters}\label{special-characters}}

There are a number of special characters that are created using a backslash (\textbackslash) before another character. These two-character codes are easily visible in the markdown because of the the backslash, where otherwise they would be difficult to notice.

\hypertarget{non-breaking-spaces}{%
\subsection{Non-breaking spaces}\label{non-breaking-spaces}}

Non-breaking spaces prevent two words from being split onto separate lines. There are a variety of cases in which they are necessary, including Scripture references. They are formed by putting a backslash prior to a regular space.

\begin{Shaded}
\begin{Highlighting}[]
\NormalTok{1\textbackslash{} Peter\textbackslash{} 1:3 }
\end{Highlighting}
\end{Shaded}

\hypertarget{line-breaks}{%
\subsection{Line breaks}\label{line-breaks}}

Sometimes (for example in poetry), you need to specify that text should start on a new line but remain part of the same paragraph. This can be accomplished by putting a single backslash at the end of a line.

\begin{Shaded}
\begin{Highlighting}[]
\NormalTok{This paragraph will continue \textbackslash{}}
\NormalTok{on the next line.}
\end{Highlighting}
\end{Shaded}

\hypertarget{backslashes}{%
\subsection{Backslashes}\label{backslashes}}

Because backslashes are are special characters, if one needs to appear for some reason in the actual text of a book, it must be `escaped' using another backslash.

\begin{Shaded}
\begin{Highlighting}[]
\NormalTok{Two backslashes }\SpecialCharTok{\textbackslash{}\textbackslash{}}\NormalTok{ will appear in the output of the book as a single backslash.}
\end{Highlighting}
\end{Shaded}

\hypertarget{poetry}{%
\section{Poetry}\label{poetry}}

Poetry should be designated as such so it can be styled differently from the rest of the text

\begin{Shaded}
\begin{Highlighting}[]
\NormalTok{::: \{.poetry\}}
\NormalTok{"Oh when a mother meets on high \textbackslash{}}
\NormalTok{The babe she lost in infancy,}

\NormalTok{Hath she not then for pains and fears, \textbackslash{}}
\NormalTok{The day of woe, the watchful night,}

\NormalTok{For all her sorrows, all her tears, \textbackslash{}}
\NormalTok{An over{-}payment of delight?"}
\NormalTok{:::}
\end{Highlighting}
\end{Shaded}

\hypertarget{foreign-languages}{%
\section{Foreign languages}\label{foreign-languages}}

Any text in a foreign language, regardless of how long or short, should be indicated as such with the appropriate two-letter language code, which can be \href{https://www.w3schools.com/tags/ref_language_codes.asp}{found here}. Here is the what the markup looks like for a couple of the most common languages you might need.

\textbf{Latin:}

\begin{Shaded}
\begin{Highlighting}[]
\CommentTok{[}\OtherTok{sacramentum}\CommentTok{]}\NormalTok{\{lang=la\}}
\end{Highlighting}
\end{Shaded}

\textbf{Greek:}

\begin{Shaded}
\begin{Highlighting}[]
\CommentTok{[}\OtherTok{μυστηριον}\CommentTok{]}\NormalTok{\{lang=el\}}
\end{Highlighting}
\end{Shaded}

\hypertarget{images}{%
\section{Images}\label{images}}

Images are likely to be rare, but when used, Pandoc's built-in functionality is not enough. Images are to be placed in a sub-folder called ``images'' and included using knitr commands:

\textbf{Centered and 50\% width image:}

\begin{Shaded}
\begin{Highlighting}[]
\InformationTok{\textasciigrave{}\textasciigrave{}\textasciigrave{}\{r, echo=FALSE, fig.align=\textquotesingle{}center\textquotesingle{}, out.width=\textquotesingle{}50\%\textquotesingle{}\}}
\InformationTok{knitr::include\_graphics("images/sepialogo.png")}
\InformationTok{\textasciigrave{}\textasciigrave{}\textasciigrave{}}
\end{Highlighting}
\end{Shaded}

Here is what it looks like when used:

\begin{center}\includegraphics[width=0.5\linewidth]{images/sepialogo} \end{center}

\hypertarget{footnotes}{%
\section{Footnotes}\label{footnotes}}

Insert the text of footnotes directly under the paragraph where the footnote occurs. This will make it much easier to find and edit the footnote in the source document.

\textbf{Simple footnote:}

\begin{Shaded}
\begin{Highlighting}[]
\NormalTok{Footnotes are often found at the end of a sentence like so.}\OtherTok{[\^{}1]}

\OtherTok{[\^{}1]: }\NormalTok{Here is the footnote\textquotesingle{}s content, in its own paragraph with a blank line above and below.}

\NormalTok{Now the main text of the book continues.}
\end{Highlighting}
\end{Shaded}

\textbf{Multi-paragraph footnote:}

\begin{Shaded}
\begin{Highlighting}[]
\NormalTok{Longer footnotes can also appear in books.}\OtherTok{[\^{}2]}

\OtherTok{[\^{}2]: }\NormalTok{Another footnote. It is possible to have multiple paragraphs in a footnote.}

\NormalTok{  Simply put two spaces before the next paragraph(s) to indicate that the footnote is continuing.}
  
\NormalTok{Here is the next paragraph of the book. It is no longer part of the footnote.}
\end{Highlighting}
\end{Shaded}


\end{document}
